\documentclass[17pt]{extarticle} % artigo que permite parágrafos com letras de tamanho diferente de 12

\usepackage[utf8]{inputenc} % pacote que permite acentos e outros caracteres especiais
\usepackage{titlesec}
%\usepackage{xcolor} % Pacote moderno para cores
\usepackage{graphicx} 
\graphicspath{{tabelas/}}

% %cores
% \definecolor{verde-escuro}{rgb}{0,0.6,0}
% \definecolor{cinza}{rgb}{0.5,0.5,0.5}
% \definecolor{mauve}{rgb}{0.58,0,0.82}
% \definecolor{azul-escuro}{HTML}{1C517B}

% Ajustando o formato e recuo das subseções
\titleformat{\subsection}[hang]{\normalfont\normalfont\bfseries}{\thesubsection}{1em}{}
\titlespacing*{\subsection}{15pt}{3.25ex plus 1ex minus .2ex}{1.5ex plus .2ex}

% Ajustando o formato e recuo das subsubseções
\titleformat{\subsubsection}[hang]{\normalfont\normalsize}{\thesubsubsection}{1em}{}[\enspace\hrulefill]
\titlespacing*{\subsubsection}{30pt}{3.25ex plus 1ex minus .2ex}{2.5ex plus .2ex}

\begin{document}

    \begin{figure}[t]
        \centering
        \includegraphics[]{unifor-logo-200x162.png}
        \label{Logo da Unifor}
    \end{figure}

    %     %      %  CAPA  %     %     %
    
    \title{
        \textbf{Questionário de pesquisa:}\\O uso de software como ferramente de revisão para o exame SPAECE
    }

    \author{
        Carolina de Souza Ribeiro (2110868)\\
        Guilherme de Farias Loureiro (2214635)\\
        Nilo José Martins Jereissati (2110887)
    }

    \date{11/08/2023}

    \maketitle

    \newpage


    %       %   PAG. INTRODUÇÃO     %           %
    \section{Introdução} 
        %1o paragrafo introdução
\paragraph{} Este formulário faz parte de um projeto de ações de cidadania do TRE em parceria com a Secretaria de Educação do Município de Baturité e objetiva sondar os professores para identificar os possíveis benefícios de um software (aplicativo) que torne revisões de conteúdo mais dinâmicas e divertidas.

%2o paragrafo introdução
\paragraph{} A priori tais revisões seriam direcionadas para o Sistema Permanente de Avaliação da Educação Básica do Ceará (SPAECE), que determina o nível escolar dos estudantes da rede pública em todo o estado. Entretanto, com o desenvolvimento e o amadurecimento do projeto seria possível abranger demais séries e disciplinas posteriormente.

%3o paragrafo introdução
\paragraph{} É plausível imaginar que no futuro o projeto possa se expandir para abranger outros contextos educacionais além do SPAECE; contudo, os dados coletados nesse formulário serão de extrema importância para a evolução da iniciativa. 
    
\paragraph{} Ao responder este formulário de forma honesta você estará colaborando para a criação de um software que poderá impactar positivamente no processo de ensino aprendizagem.
    \newpage


    %    %  PAG. OBJ. ESPECIFICO  %   %

    \section{Objetivos}
        \begin{enumerate} % Lista de itens
            \item 
            \item $\rightarrow$ ;
            \item $\rightarrow$ ;
            \item $\rightarrow$ ************************;
            \item $\rightarrow$ ************************;
            \item $\rightarrow$ ************************;
        \end{enumerate}

    \section{Metodologia}

        %informações sobre a analise
    \   \begin{itemize} % Lista de itens
            %formula do n
            \item Fórmula usada: \( n = \frac{N \cdot (1 / (0.02 \cdot 0.05))}{(1 / (0.02 \cdot 0.05)) + N} \);

            \item Margem de erro: 2\%;
            \item Tamanho da amostra: 200;
            \item Valor do \( n \): 185;
        \end{itemize} %fim da lista de itens

    %fim da pagina objetivo específico
    \newpage


    %    %  %  QUESTIONÁRIO  %  %    %
    \section{Questionário}

    %seções do google forms dentro de subseções

        \subsection{Informações pessoais e profissionais}

            %pergunta
            \subsubsection{Há quantos anos você trabalha como docente?\enspace\hrulefill}

            %pergunta
            \subsubsection{Em quais séries do ensino fundamental você ministra aulas?}

            %espaço entre uma pergunta e outra
            \vspace{3mm} 


            % Continuação do questionário...

    %fim da pagina
    \newpage


%        %   % CONCLUSÃO %   %         %

\section{Conclusão}
.............................

\end{document}